\subsection*{Интернет}
	Интернет - всемирная глобальная компьютерная сеть. Интернет объединяет в себе тысячи компьютерных сетей мира. Отдельный пользователь, не являющийся абонентом компьютерных сетей, может подключиться к Интернету через ближайший узловой центр. В настоящее время число пользователей Интернетом превышает 2 миллиарда человек. Хакер - чрезвычайно квалифицированный ИТ- специалист, человек, который понимает самые основы работы компьютерных систем.

\subsection*{Хакер}
	Изначально хакерами называли программистов, которые исправляли ошибки в программном обеспечении каким-либо быстрым и далеко не всегда элегантным или профессиональным способом; такие правки ассоциировались с «топорной работой» из-за их грубости, отсюда и произошло название «хакер».

\subsection*{Система}
	Фрикинг - сленговое выражение, означающее взлом телефонных автоматов и сетей, обычно с целью осуществления бесплатных звонков. Людей, специализирующихся на фрикинге, называют фрикерами. Это же название применяют к людям, использующим в своих неправомерных действиях телефон с целью оказать психологическое воздействие на конечного абонента. В последнее время под фрикингом стали подразумевать различный взлом электронных систем.

